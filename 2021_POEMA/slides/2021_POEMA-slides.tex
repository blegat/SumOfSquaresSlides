\documentclass{beamer}
\usetheme[block=fill]{metropolis}
%\setbeamersize{text margin left=.2cm,text margin right=.2cm}
\usepackage{graphicx}
\graphicspath{{../../images/}}
%\usepackage[french]{babel}
%\usepackage{listings}
%\usepackage{lipsum}
\usepackage{boolexpr}
\usepackage{kpfonts}
\usepackage{caption}
\usepackage{wrapfig}
%\usepackage{chngcntr}
\usepackage[labelformat=empty]{caption}
\usepackage[official]{eurosym}

\usepackage{hyperref}
\usepackage[charsperline=55,theme=darkbeamer]{jlcode}

\newcommand{\induced}{\rho}
\newcommand{\induceda}{\induced^{(1)}}
\newcommand{\inducedb}{\induced^{(2)}}

\newcommand{\citefoot}[1]{\footnote{\tiny #1}}

\usepackage{minted}

\usepackage{siunitx}

% http://tex.stackexchange.com/questions/114830/how-can-i-use-lvert-and-rvert-norm-symbols-x-with-the-iwona-math-font
\usepackage[math]{iwona}
\usepackage{scalerel}
\def\lVert{\mid\!\mid}
\def\rVert{\mid\!\mid}

\usepackage[normalem]{ulem}
%\newcommand{\Adj}{\mathbf{A}}
\usepackage{mathtools}

\usepackage{../../custom}
\usepackage{amsfonts}
%\newcommand{\jsrcodepath}{../../code}
%\usepackage{jsr}

\newcommand{\expe}[2]{\la #1, #2 \ra}

\usepackage{framed}

%\usepackage{mathtools,xparse}
%\DeclarePairedDelimiter{\norm}{\lVert}{\rVert}
\newcommand\Wider[2][3em]{%
\makebox[\linewidth][c]{%
  \begin{minipage}{\dimexpr\textwidth+#1\relax}
  \raggedright#2
  \end{minipage}%
  }%
}

\newcommand{\aeur}{\alpha_\text{\euro}}
\newcommand{\adol}{\alpha_\$}
\newcommand{\apou}{\alpha_\text{\pounds}}

\title{Sum-of-Squares optimization with JuMP in Julia}
\date{September, 2021}
\author{Beno\^it Legat}
\institute{POEMA Learning Week 2}
%\institute{UCLouvain\\
%           Massachusetts Institute of Technology (MIT)\\
%           Rice University\\
%           Los Alamos National Laboratory (LANL)}

% https://tex.stackexchange.com/questions/426088/texlive-pretest-2018-beamer-and-subfig-collide
\makeatletter
\let\@@magyar@captionfix\relax
\makeatother
\begin{document}
  \maketitle

  \begin{frame}
    \tableofcontents
  \end{frame}

  \section{Background}

\begin{frame}{Nonnegative polynomial into sum of squares (SOS)}
  \begin{tikzpicture}
    \draw[->, bend left=20] (.6, 1.6) node[above] {$(x^3, x^2, x)$} to (.9, 1.35);
    \draw[->, bend right=30] (2.1, 2) node[right] {\alert{\emph{not} unique}} to (1.55, 1.35);
    \node at (-.1, 1.2) {$p(x)$};
    \node at (.5, 1.2) {$=$};
    \node at (1.3, 1.2) {$X^\Tr Q X$};
    \node at (-1.5, 0) {$x^6 - 2x^4 + 2x^2$% (\citefoot{\gatpar{}: Example~5.1})
    };
    \node at (.5, 0) {$=$};
    \node at (2.4, 0) {$X^\Tr \begin{pmatrix}1 & 0 & -1\\\cdot & 0 & 0\\ \cdot & \cdot & 2\end{pmatrix} X$};
    \node at (-1, -1.5) {$p(x) \geq 0$ $\forall x$};
    \node at (1.5, -1.5) {$Q \succeq 0$};
    \node at (.5, -1.5) {$\Longleftarrow$};
    \draw[->] (2.5, -1) to node[right] {cholesky} (2.5, -2.5);
    \node at (-2.2, -3.5) {$x^2 + (-x^3 + x)^2$};
    \draw[->] (.2, -3.5) to (-.6, -3.5);
    \node at (3.3, -3.5) {$X^\Tr \begin{pmatrix}0 & 0 & 1\\-1 & 0 & 1\end{pmatrix}^\Tr \begin{pmatrix}0 & 0 & 1\\-1 & 0 & 1\end{pmatrix} X$};
  \end{tikzpicture}

      %\noindent {\tiny Gatermann, Parrilo, \emph{{S}ymmetry groups, semidefinite programs, and sums of squares}, Journal of Pure and Applied Algebra, 2004: Example~5.1}
\end{frame}

\begin{frame}
  \frametitle{Polynomial optimization in Julia}
  \begin{itemize}
    \item \url{ahmadreza-marandi/Polyopt.jl}%\citefoot{A. Marandi, E. de Klerk, J. Dahl (2020), “Solving sparse polynomial optimization problems with chordal structure using the sparse, bounded-degree sum-of-squares hierarchy,” Disc App Math, Volume 275, pp. 95--110. \url{doi:10.1016/j.dam.2017.12.015}.}.
    \item \url{rdeits/Mayday.jl}
    \item \url{ameliaperry/SumOfSquaresOptimization.jl}
    \item \url{JulieSliwak/MathProgComplex.jl}%\citefoot{Gilles Bareilles, Manuel Ruiz, Julie Sliwak, ``\emph{A Julia JuMP-based module for polynomial optimization with complex variables applied to ACOPF}'', JuMP-dev workshop 2018}.
    \item \url{jump-dev/PolyJuMP.jl}, \url{jump-dev/SumOfSquares.jl}
    \item \url{lanl-ansi/MomentOpt.jl}
    \item \url{tweisser/PolyPowerModels}, \url{tweisser/PolyModels}
    \item \url{wangjie212/TSSOS}, \url{wangjie212/NCTSSOS}
    \item \url{wangjie212/SONCSOCP}
    \item \url{blegat/CondensedMatterSOS.jl}
  \end{itemize}
\end{frame}
\begin{frame}
  \frametitle{History}
  \begin{itemize}
    \item Started in August 2016: \texttt{SOSTOOLS} demos.
    \item October 2016: Amelia Perry, Joey Huchette, Robin Deits and Chris Coey \alert{join},
      package \alert{split} into \texttt{MultivariatePolynomials}, \texttt{PolyJuMP} and \texttt{SumOfSquares}.
    \item April 2017, \texttt{DynamicPolynomials}, \texttt{TypedPolynomials},
    \item August 2017 \texttt{MultivariateMoments} and \texttt{SemialgebraicSets} \alert{factored out} of \texttt{SumOfSquares}.
    \item 2017--2019 \texttt{MathOptInterface} (MOI) $\to$ \alert{rewrite} of \texttt{JuMP}, \texttt{PolyJuMP} and \texttt{SumOfSquares}.
    \item 2019--2020 \texttt{MutableArithmetics} for \alert{performance}.
    \item 2019--2020 Tillmann Tweisser \alert{joins}, develop \texttt{MomentOpt}.
    \item 2020--2021 Marek Kaluba \alert{joins}, develop \texttt{SymbolicWedderburn} for symmetry reduction.
  \end{itemize}
\end{frame}

\begin{frame}{Overview}
\Wider[4em]{
  \begin{center}
    \begin{tikzpicture}[scale=0.7]
      \draw[rounded corners=5pt, fill=aurore!70] (-2.1, 5.5) rectangle (2.1, 6.5);
      \node at (0, 6) {\jlpkg{MomentOpt}};
      \draw[rounded corners=5pt, fill=frambo!50] (-2.5, 3.5) rectangle (2.5, 4.5);
      \node at (0, 4) {\jlpkg{SumOfSquares}};
      \draw[rounded corners=5pt, fill=orange!50] (-1.9, 1.5) rectangle (1.9, 2.5);
      \node at (0, 2) {\jlpkg{PolyJuMP}};
      \node at (0, 0) {\includegraphics[scale=0.2]{JuMP.png}};
      \draw[->, line width=1pt] (0, 5.5) to (0, 4.5);
      \draw[->, line width=1pt] (0, 3.5) to (0, 2.5);
      \draw[->, line width=1pt] (0, 1.5) to (0, 0.5);
      %\draw[rounded corners=5pt] (-3, -6.5) rectangle (3, -1.5);
      %\node[rotate=90] at (-2.7, -4) {\jlpkg{MathOptInterface}};
      %\node[rotate=-90] at (2.7, -4) {MOI};
      \draw[rounded corners=5pt, fill=lichen!70] (0.6, -2.5) rectangle (3.4, -1.5);
      \node at (2, -2) {SOSBridge};
      %\draw[rounded corners=5pt, fill=aurore!70] (-1, -4.5) rectangle (1, -5.5);
      %\node at (0, -5) {Caching};
      %\draw[->, line width=1pt] (0, -3.5) to (0, -4.5);
      \draw[->, line width=1pt] (-0.5, -0.8) to (-3, -3.5);
      \draw[->, line width=1pt] (0.5, -0.8) to (0.9, -1.5);
      \draw[->, line width=1pt] (3, -2.5) to (3.5, -3.5);
      \draw[rounded corners=5pt, fill=canard!60] (-5.4, -4.5) rectangle (-2.6, -3.5);
      \node at (-4, -4) {SOS solver};
      \draw[rounded corners=5pt, fill=canard!60] (2.5, -4.5) rectangle (5.5, -3.5);
      \node at (4, -4) {SDP solver};
    \end{tikzpicture}
  \end{center}
}
\end{frame}


  \section{Examples}
  \begin{frame}
    \frametitle{Polynomial Optimization}
    \url{https://jump.dev/SumOfSquares.jl/stable/generated/Polynomial\%20Optimization/polynomial_optimization/}
  \end{frame}
  \begin{frame}
    \frametitle{Noncommutative variables}
    \url{https://jump.dev/SumOfSquares.jl/stable/generated/Noncommutative\%20and\%20Hermitian/noncommutative_variables/}
  \end{frame}
  \begin{frame}
    \frametitle{Symmetry reduction}
    \begin{itemize}
      \item \url{https://jump.dev/SumOfSquares.jl/stable/generated/Symmetry/permutation_symmetry/}
      \item \url{https://jump.dev/SumOfSquares.jl/stable/generated/Symmetry/even_reduction/}
      \item \url{https://jump.dev/SumOfSquares.jl/stable/generated/Symmetry/dihedral_symmetry_of_the_robinson_form/}
    \end{itemize}
  \end{frame}

  \section{Extensions}
  \begin{frame}
    \frametitle{New polynomial implementation}
    \url{https://blegat.github.io/CondensedMatterSOS.jl/dev/generated/Ising/}
  \end{frame}
  \begin{frame}
    \frametitle{New certificate}
      \url{https://jump.dev/SumOfSquares.jl/stable/generated/Extension/certificate/}

      If certificate not \alert{flexible} enough:
        define new MathOptInterface \alert{bridge}.
  \end{frame}
  \begin{frame}
    \frametitle{New solver}
    \url{https://jump.dev/SumOfSquares.jl/stable/generated/Extension/univariate_solver/}
  \end{frame}
  \begin{frame}[fragile]
    \frametitle{PolyJuMP extension}
    What does this \alert{means} ?
\begin{lstlisting}[language = Julia]
@polyvar x y
@constraint(model, x^4 + y^4 >= x * y - 1)
\end{lstlisting}
    PolyJuMP needs a \alert{polymodule} (shortcut: \lstinline|SOSModel|):
\begin{lstlisting}[language = Julia]
model = Model()
setpolymodule!(model, SumOfSquares)
\end{lstlisting}
    With \jlpkg{SumOfSquares}, it is \alert{interpreted} as:
%\begin{lstlisting}[language = Julia]
\begin{jllisting}[language=julia, style=jlcodestyle]
@constraint(model, x^4 + y^4 - x * y + 1 in SOSCone())
\end{jllisting}
%\end{lstlisting}
  \end{frame}

  \begin{frame}
    \frametitle{New bases}
    MultivariateBases provides:
    \begin{itemize}
      \item \texttt{FixedPolynomialBasis}: \alert{Any} vector of polynomials.
      \item \alert{Monomial} bases: \texttt{MonomialBasis} and \texttt{ScaledMonomialBasis}.
      \item \alert{Orthogonal} bases:
        \begin{itemize}
          \item Hermite bases: \texttt{ProbabilistsHermiteBasis} and \texttt{PhysicistsHermiteBasis}.
          \item \texttt{LaguerreBasis}.
          \item Gegenbauer bases:
            \begin{itemize}
              \item LegendreBasis.
              \item Chebyshev bases: \texttt{ChebyshevBasisFirstKind} and \texttt{ChebyshevBasisSecondKind}
            \end{itemize}
        \end{itemize}
    \end{itemize}
    \alert{Any} other basis implementing the MultivariateBases \alert{API} can be used with SumOfSquares!
  \end{frame}

  \begin{frame}
    \frametitle{New algebraic sets}
    See \url{https://jump.dev/SumOfSquares.jl/stable/generated/Extension/hypercube/}.
  \end{frame}

  \begin{frame}
    \frametitle{Conclusion}
    \begin{itemize}
      \item \jlpkg{MathOptInterface} (MOI) is extensible by design:
        Defining new set and bridges is the way to go.
      \item \jlpkg{JuMP} macros are designed for extensions as well:
      \item \jlpkg{PolyJuMP} extends JuMP macros for polynomials,
        defines MOI set and bridges for polynomial equality in algebraic set.
      \item \jlpkg{SumOfSquares} is a PolyJuMP's polymodule,
        defines MOI set and bridges (\alert{parametrized} by \alert{certificates}) for SOS constraints in basic semialgebraic set.
    \end{itemize}
  \end{frame}

\end{document}
