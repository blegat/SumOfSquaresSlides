\begin{frame}{Nonnegative polynomial into sum of squares}
  \begin{tikzpicture}
    \draw[->, bend left=30] (-1, 1.6) node[left] {$(x_1, x_2, x_3)$} to (-.1, 1.3);
    \draw[->, bend left=20] (.6, 1.6) node[above] {$(x_1, x_1x_2, x_2)$} to (.9, 1.35);
    \draw[->, bend right=30] (2.1, 2) node[right] {\alert{\emph{not} unique}} to (1.55, 1.35);
    \node at (-.1, 1.2) {$p(x)$};
    \node at (.5, 1.2) {$=$};
    \node at (1.3, 1.2) {$X^\Tr Q X$};
    \node at (-2.5, 0) {$x_1^2 + 2x_1^2x_2 + 5x_1^2x_2^2 + 4x_1x_2^2 + x_2^2$};
    \node at (.5, 0) {$=$};
    \node at (2.3, 0) {$X^\Tr \begin{pmatrix}1 & 1 & 0\\1 & 5 & 2\\ 0 & 2 & 1\end{pmatrix} X$};
    \node at (-1, -1.5) {$p(x) \geq 0$ $\forall x$};
    \node at (1.5, -1.5) {$Q \succeq 0$};
    \node at (.5, -1.5) {$\Longleftarrow$};
    \draw[->] (2.5, -1) to node[right] {cholesky} (2.5, -2.5);
    \node at (-3, -3.5) {$(x_1 + x_1x_2)^2 + (2x_1x_2 + x_2)^2$};
    \draw[->] (.2, -3.5) to (-.6, -3.5);
    \node at (3, -3.5) {$X^\Tr \begin{pmatrix}1 & 1 & 0\\0 & 2 & 1\end{pmatrix}^\Tr \begin{pmatrix}1 & 1 & 0\\0 & 2 & 1\end{pmatrix} X$};
  \end{tikzpicture}
\end{frame}

\begin{frame}{Schur lemma}
  \noindent$(X_i)_i\subseteq \R^{n \times n}$ \emph{reducible} if $\exists \{0\} \subset \mathcal{V} \subseteq \R^n$ s.t. $\forall i, X_i \mathcal{V} \subseteq \mathcal{V}$

  \noindent$(X_i)_i\subseteq \R^{n \times n}$ \emph{equivalent} to $(Y_i)_i\subseteq \R^{n \times n}$ if $\exists T$ s.t. $X_i = TY_iT^{-1}$

  If $X, Y$ are irreducible and inequivalent, then
  $$
  %\begin{bmatrix}
  %  Q_{11} & Q_{12}\\
  %  Q_{21} & Q_{22}
  %\end{bmatrix}
  Q
  \begin{bmatrix}
    X & 0\\
    0 & Y
  \end{bmatrix}
    =
  \begin{bmatrix}
    X & 0\\
    0 & Y
  \end{bmatrix}
  Q
  %\begin{bmatrix}
  %  Q_{11} & Q_{12}\\
  %  Q_{21} & Q_{22}
  %\end{bmatrix}
  \quad
  \Rightarrow
  \quad
  \exists c_1, c_2 \text{ s.t. }
  Q =
  \begin{bmatrix}
    c_1 I & 0\\
    0 & c_2 I
  \end{bmatrix}
   Q_{11} = q_1 I, Q_{22} = q_2 I,
   Q_{21} = Q_{12} = 0.
  $$

  If $X$ is irreducible, then
  $$
  %\begin{bmatrix}
  %  Q_{11} & Q_{12}\\
  %  Q_{21} & Q_{22}
  %\end{bmatrix}
  Q
  \begin{bmatrix}
    X & 0\\
    0 & X
  \end{bmatrix}
    =
  \begin{bmatrix}
    X & 0\\
    0 & X
  \end{bmatrix}
  Q
  %\begin{bmatrix}
  %  Q_{11} & Q_{12}\\
  %  Q_{21} & Q_{22}
  %\end{bmatrix}
  \quad
  \Rightarrow
  \quad
  \exists C \in \SymK \text{ s.t. }
  Q = C \otimes I =
  \begin{bmatrix}
    c_{1,1}I & c_{1,2}I\\
    c_{1,2}I & c_{2,2}I
  \end{bmatrix}
   Q_{11} = q_1 I, Q_{22} = q_2 I,
   Q_{21} = Q_{12} = 0.
  $$

  \noindent {\tiny Sagan, \emph{The symmetric group: representations, combinatorial algorithms, and symmetric functions}, Springer Science \& Business Media, 2001: Definition~1.4.5, Definition~1.6.2, Example~1.7.2, Example~1.7.3 (adapted from $\mathcal{C}^{n \times n}$ to symmetric matrices $\SymK \subseteq \mathcal{R}^{n \times n}$)}
\end{frame}
