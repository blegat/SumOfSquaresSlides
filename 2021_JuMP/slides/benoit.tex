\begin{frame}{Nonnegative polynomial into sum of squares}
  \begin{tikzpicture}
    \draw[->, bend left=20] (.6, 1.6) node[above] {$(x^3, x^2, x)$} to (.9, 1.35);
    \draw[->, bend right=30] (2.1, 2) node[right] {\alert{\emph{not} unique}} to (1.55, 1.35);
    \node at (-.1, 1.2) {$p(x)$};
    \node at (.5, 1.2) {$=$};
    \node at (1.3, 1.2) {$X^\Tr Q X$};
    \node at (-1.5, 0) {$x^6 - 2x^4 + 2x^2$};
    \node at (.5, 0) {$=$};
    \node at (2.4, 0) {$X^\Tr \begin{pmatrix}1 & 0 & -1\\\cdot & 0 & 0\\ \cdot & \cdot & 2\end{pmatrix} X$};
    \node at (-1, -1.5) {$p(x) \geq 0$ $\forall x$};
    \node at (1.5, -1.5) {$Q \succeq 0$};
    \node at (.5, -1.5) {$\Longleftarrow$};
    \draw[->] (2.5, -1) to node[right] {cholesky} (2.5, -2.5);
    \node at (-2.2, -3.5) {$x^2 + (-x^3 + x)^2$};
    \draw[->] (.2, -3.5) to (-.6, -3.5);
    \node at (3.3, -3.5) {$X^\Tr \begin{pmatrix}0 & 0 & 1\\-1 & 0 & 1\end{pmatrix}^\Tr \begin{pmatrix}0 & 0 & 1\\-1 & 0 & 1\end{pmatrix} X$};
  \end{tikzpicture}

      \noindent {\tiny Gatermann, Parrilo, \emph{{S}ymmetry groups, semidefinite programs, and sums of squares}, Journal of Pure and Applied Algebra, 2004: Example~5.1}
\end{frame}

\begin{frame}{Semidefinite program}
%  \begin{columns}
    %\begin{column}{0.48\textwidth}
      \begin{equation*}
        \label{eq:sdp_standard}
        \begin{array}{rl}
          \min\limits_{X \in \SymK} & \la C, Q \ra\\
          \text{subject to:} & \la A_i, Q \ra = b_i, \quad i=1,2,\ldots,m\\
          & Q \succeq 0.
        \end{array}
      \end{equation*}
    %\end{column}
    %\begin{column}{0.48\textwidth}
      Is $x^6 - 2x^4 + 2x^2$ a sum of squares ?
      $$
      X =
      \begin{bmatrix}
        x^3\\
        x^2\\
        x
      \end{bmatrix},
      XX^\Tr =
      \begin{bmatrix}
        x^6 & x^5 & x^4\\
        \cdot & x^4 & x^3\\
        \cdot & \cdot & x^2
      \end{bmatrix}
      A_{x^6} =
      \begin{bmatrix}
        1 & 0 & 0\\
        \cdot & 0 & 0\\
        \cdot & \cdot & 0
      \end{bmatrix},
      A_{x^5} =
      \begin{bmatrix}
        0 & 1 & 0\\
        \cdot & 0 & 0\\
        \cdot & \cdot & 0
      \end{bmatrix},
      $$
      $$
      A_{x^4} =
      \begin{bmatrix}
        0 & 0 & 1\\
        \cdot & 1 & 0\\
        \cdot & \cdot & 0
      \end{bmatrix},
      A_{x^3} =
      \begin{bmatrix}
        0 & 0 & 0\\
        \cdot & 0 & 1\\
        \cdot & \cdot & 0
      \end{bmatrix}
      A_{x^2} =
      \begin{bmatrix}
        0 & 0 & 0\\
        \cdot & 0 & 0\\
        \cdot & \cdot & 1
      \end{bmatrix}
      $$
    %\end{column}
%  \end{columns}

      \noindent {\tiny Gatermann, Parrilo, \emph{{S}ymmetry groups, semidefinite programs, and sums of squares}, Journal of Pure and Applied Algebra, 2004: Example~5.1}
\end{frame}

\begin{frame}{Schur lemma}
  \noindent$(X_i)_i\subseteq \R^{n \times n}$ \emph{reducible} if $\exists \{0\} \subset \mathcal{V} \subseteq \R^n$ s.t. $\forall i, X_i \mathcal{V} \subseteq \mathcal{V}$

  \noindent$(X_i)_i\subseteq \R^{n \times n}$ \emph{equivalent} to $(Y_i)_i\subseteq \R^{n \times n}$ if $\exists T$ s.t. $X_i = TY_iT^{-1}$

  If $X, Y$ are irreducible and inequivalent, then
  $$
  %\begin{bmatrix}
  %  Q_{11} & Q_{12}\\
  %  Q_{21} & Q_{22}
  %\end{bmatrix}
  Q
  \begin{bmatrix}
    X & 0\\
    0 & Y
  \end{bmatrix}
    =
  \begin{bmatrix}
    X & 0\\
    0 & Y
  \end{bmatrix}
  Q
  %\begin{bmatrix}
  %  Q_{11} & Q_{12}\\
  %  Q_{21} & Q_{22}
  %\end{bmatrix}
  %\quad
  \Rightarrow
  %\quad
  \exists c_1, c_2 \text{ s.t. }
  Q =
  \begin{bmatrix}
    c_1 I & 0\\
    0 & c_2 I
  \end{bmatrix}
   %Q_{11} = q_1 I, Q_{22} = q_2 I,
   %Q_{21} = Q_{12} = 0.
  $$

  If $X$ is irreducible, then (permuting rows/columns give
  $I \otimes C$)
  $$
  %\begin{bmatrix}
  %  Q_{11} & Q_{12}\\
  %  Q_{21} & Q_{22}
  %\end{bmatrix}
  Q
  \begin{bmatrix}
    X & 0\\
    0 & X
  \end{bmatrix}
    =
  \begin{bmatrix}
    X & 0\\
    0 & X
  \end{bmatrix}
  Q
  %\begin{bmatrix}
  %  Q_{11} & Q_{12}\\
  %  Q_{21} & Q_{22}
  %\end{bmatrix}
  %\quad
  \Rightarrow
  %\quad
  \exists C \in \SymK \text{ s.t. }
  Q = C \otimes I =
  \begin{bmatrix}
    c_{1,1}I & c_{1,2}I\\
    c_{1,2}I & c_{2,2}I
  \end{bmatrix}
   %Q_{11} = q_1 I, Q_{22} = q_2 I,
   %Q_{21} = Q_{12} = 0.
  $$

  \noindent {\tiny Sagan, \emph{The symmetric group}, Springer Science \& Business Media, 2001: Definition~1.4.5, Definition~1.6.2, Example~1.7.2, Example~1.7.3 (adapted from $\mathcal{C}^{n \times n}$ to symmetric matrices $\SymK \subseteq \mathcal{R}^{n \times n}$)}
\end{frame}
