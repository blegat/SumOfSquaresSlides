\documentclass{beamer}
\usetheme{metropolis}
%\setbeamersize{text margin left=.2cm,text margin right=.2cm}
\usepackage{graphicx}
% \usepackage[french]{babel}
%\usepackage{listings}
%\usepackage{lipsum}
\usepackage{boolexpr}
\usepackage{kpfonts}
\usepackage{caption}
\usepackage{wrapfig}
\usepackage{tkz-graph}
\usepackage{tikz}
\usepackage{chngcntr}
\usepackage[labelformat=empty]{caption}
\usepackage[official]{eurosym}

\usepackage{minted}

\usepackage{siunitx}

% http://tex.stackexchange.com/questions/114830/how-can-i-use-lvert-and-rvert-norm-symbols-x-with-the-iwona-math-font
\usepackage[math]{iwona}
\usepackage{scalerel}
\def\lVert{\mid\!\mid}
\def\rVert{\mid\!\mid}

\usepackage[normalem]{ulem}
\newcommand{\Adj}{\mathbf{A}}
\usepackage{mathtools}

\usepackage{custom}
\usepackage{amsfonts}
\newcommand{\jsrcodepath}{../../code}
\usepackage{jsr}

\newcommand{\expe}[2]{\la #1, #2 \ra}

\usepackage{framed}

%\usepackage{mathtools,xparse}
%\DeclarePairedDelimiter{\norm}{\lVert}{\rVert}
\newcommand\Wider[2][3em]{%
\makebox[\linewidth][c]{%
  \begin{minipage}{\dimexpr\textwidth+#1\relax}
  \raggedright#2
  \end{minipage}%
  }%
}

\newcommand{\aeur}{\alpha_\text{\euro}}
\newcommand{\adol}{\alpha_\$}
\newcommand{\apou}{\alpha_\text{\pounds}}

\title{Sum-of-squares optimization in Julia}
\date{}
\author{Beno\^it Legat (UCL)\\\emph{Joint Work with:}\\Chris Coey, Robin Deits, Joey Huchette and Amelia Perry (MIT)}
\institute{Universit\'e catholique de Louvain (UCL)\\
           Massachusetts Institute of Technology (MIT)}

\begin{document}
  \maketitle
  \begin{frame}{Nonnegative quadratic forms into sum of squares}
    \begin{tikzpicture}
      \draw[->, bend left=30] (-1, 1.6) node[left] {$(x_1, x_2, x_3)$} to (-.1, 1.3);
      \draw[->, bend left=30] (-1, 1.6) to (.9, 1.25);
      \draw[->, bend right=30] (2.1, 2) node[right] {\alert{unique}} to (1.55, 1.35);
      \node at (-.2, 1.2) {$p(x)$};
      \node at (.5, 1.2) {$=$};
      \node at (1.3, 1.2) {$x^\Tr Q x$};
      \node at (-2.4, 0) {$x_1^2 + 2x_1x_2 + 5x_2^2 + 4x_2x_3 + x_3^2$};
      \node at (.5, 0) {$=$};
      \node at (2.4, 0) {$x^\Tr \begin{pmatrix}1 & 1 & 0\\1 & 5 & 2\\ 0 & 2 & 1\end{pmatrix} x$};
      \node at (-1, -1.5) {$p(x) \geq 0$ $\forall x$};
      \node at (1.5, -1.5) {$Q \succeq 0$};
      \node at (.5, -1.5) {$\Longleftrightarrow$};
      \draw[->] (2.5, -1) to node[right] {cholesky} (2.5, -2.5);
      \node at (-3, -3.5) {$(x_1 + x_2)^2 + (2x_2 + x_3)^2$};
      \draw[->] (.2, -3.5) to (-.8, -3.5);
      \node at (3, -3.5) {$x^\Tr \begin{pmatrix}1 & 1 & 0\\0 & 2 & 1\end{pmatrix}^\Tr \begin{pmatrix}1 & 1 & 0\\0 & 2 & 1\end{pmatrix} x$};
    \end{tikzpicture}
  \end{frame}
  \begin{frame}{Nonnegative polynomial into sum of squares}
    \begin{tikzpicture}
      \draw[->, bend left=30] (-1, 1.6) node[left] {$(x_1, x_2, x_3)$} to (-.1, 1.3);
      \draw[->, bend left=20] (.6, 1.6) node[above] {$(x_1, x_1x_2, x_2)$} to (.9, 1.35);
      \draw[->, bend right=30] (2.1, 2) node[right] {\alert{\emph{not} unique}} to (1.55, 1.35);
      \node at (-.2, 1.2) {$p(x)$};
      \node at (.5, 1.2) {$=$};
      \node at (1.3, 1.2) {$X^\Tr Q X$};
      \node at (-2.4, 0) {$x_1^2 + 2x_1^2x_2 + 5x_1^2x_2^2 + 4x_1x_2^2 + x_2^2$};
      \node at (.5, 0) {$=$};
      \node at (2.4, 0) {$X^\Tr \begin{pmatrix}1 & 1 & 0\\1 & 5 & 2\\ 0 & 2 & 1\end{pmatrix} X$};
      \node at (-1, -1.5) {$p(x) \geq 0$ $\forall x$};
      \node at (1.5, -1.5) {$Q \succeq 0$};
      \node at (.5, -1.5) {$\Longleftarrow$};
      \draw[->] (2.5, -1) to node[right] {cholesky} (2.5, -2.5);
      \node at (-3, -3.5) {$(x_1 + x_1x_2)^2 + (2x_1x_2 + x_2)^2$};
      \draw[->] (.2, -3.5) to (-.8, -3.5);
      \node at (3, -3.5) {$X^\Tr \begin{pmatrix}1 & 1 & 0\\0 & 2 & 1\end{pmatrix}^\Tr \begin{pmatrix}1 & 1 & 0\\0 & 2 & 1\end{pmatrix} X$};
    \end{tikzpicture}
  \end{frame}
  \begin{frame}{When is nonnegativity equivalent to sum of squares ?} 
    \begin{block}{Hilbert 1888}
      Nonnegativity of $p(x)$ of $n$ variables and degree $2d$ is equivalent to sum of squares in the following three cases:
      \begin{itemize}
        \item $n = 1$ : Univariate polynomials
        \item $2d = 2$ : Quadratic polynomials
        \item $n = 2$, $2d = 4$ : Bivariate quartics
      \end{itemize}
    \end{block}
    \begin{block}{Motzkin 1967}
      First explicit example:
      \[ x_1^4x_2^2 + x_1^2x_2^4 + 1 - 3x_1^2x_2^2 \geq 0 \quad \forall x \]
      but is \alert{not} a sum of squares.
    \end{block}
  \end{frame}
  \begin{frame}{Sum-of-Squares cone}
    \begin{block}{Nonnegative orthant $\R^n_+ \subset \R^n$}
      Proper cone and self-dual with scalar product
      \[ \la a, b \ra = b^\Tr a. \]
    \end{block}
    \vspace{-1.5em}
    \begin{block}{Semidefinite cone $\Psd \subset \SymK$}
      Proper cone and self-dual with scalar product
      \[ \la A, B \ra = \trace(B A). \]
    \end{block}
    \vspace{-1.5em}
    \begin{block}{Sum-of-Squares cone $\Sos[n, 2d] \subset \R[x]_{n, 2d}$}
      Proper cone and dual $\Sos[2, 2d]^*$ with scalar product
      \[ \la \mu, p \ra = \int_{\R^n} p(x) \mu(\dif x). \]
      is the cone of \emph{pseudo measures}.
    \end{block}
  \end{frame}
  \begin{frame}{What is Sum-of-squares programming ?}
    %Sum-of-squares programming is a generalization of Semidefinite programming:
    \begin{block}{Linear Programming}
      \vspace{-1em}
      \begin{align*}
        \mini_{x \in \R^n}\quad & \la c, x \ra & \maxi_{y \in \R^n} \quad & \la b, y \ra\\
        \subtoq & Ax = b & \subtoq & A^\Tr y \leq c\\
        & x \geq 0
      \end{align*}
    \end{block}
    \vspace{-2em}
    \begin{block}{Semidefinite Programming}
      \vspace{-1em}
      \begin{align*}
        \mini_{Q \in \SymK} \quad & \la C, Q \ra & \maxi_{y \in \R^n} \quad & \la b, y \ra\\
        \subtoq & \la A_i, Q \ra = b_i & \subtoq & \sum_i A_i y_i \preceq C\\
          & Q \succeq 0
      \end{align*}
      %Linear Programming :
      $A_i = \Diag(a_i), C = \Diag(c), Q = \Diag(x)$
    \end{block}
  \end{frame}
  \begin{frame}{What is Sum-of-squares programming ?}
    \begin{block}{Semidefinite Programming}
      \vspace{-1em}
      \begin{align*}
        \mini_{Q \in \SymK} \quad & \la C, Q \ra & \maxi_{y \in \R^n} \quad & \la b, y \ra\\
        \subtoq & \la A_i, Q \ra = b_i & \subtoq & \sum_i A_i y_i \preceq C\\
          & Q \succeq 0
      \end{align*}
    \end{block}
    \vspace{-2em}
    \begin{block}{Sum-of-squares Programming}
      \vspace{-1em}
      \begin{align*}
        \mini_{p \in \R[x]_{n, 2d}} \la \nu, p \ra && \max \la b, y \ra\\
        \la \mu_i, p \ra & = b_i & \sum_i \mu_i y_i & \preceq \nu\\
        p & \succeq 0
      \end{align*}
    \end{block}
    \vspace{-1em}
    %Semidefinite Programming:\\
    $(A_k)_{ij} = \la \mu_k, x_ix_j \ra, C_{ij} = \la \nu, x_ix_j \ra, p(x) = x^\Tr Q x$
  \end{frame}
  \begin{frame}{Sum of Squares in Julia : A joint effort}
    \begin{center}
      \scriptsize
      \begin{tikzpicture}
        \SetVertexNormal[Shape=rectangle,FillColor = yellow!50]
        \draw[rounded corners=6pt] (-6.4, 2.5) rectangle (4, 3.5);
        \Vertex[x=-4,y=3,L={\jlpkg{SumOfSquaresOptimization}}]{SOSO}
        \Vertex[x= 0,y=3,L={\jlpkg{SumOfSquares}}]{SOSold}
        \Vertex[x= 3,y=3,L={\jlpkg{MayDay}}]{MD}

        %\draw[->, bend left=20] (-4, 2.5) to (-1, 0.5);
        %\draw[->, bend left=20] (-4, 2.5) to (-1, 0.5);
        \draw[->] (-4, 2.5) .. controls (-1, 2) and (-1, 1.5) .. (-1, 0.5);
        \draw[->] ( 3, 2.5) .. controls (-1, 2) and (-1, 1.5) .. (-1, 0.5);
        \draw[->] ( 0, 2.5) .. controls (-1, 2) and (-1, 1.5) .. (-1, 0.5);

        \draw[rounded corners=6pt] (-6.6, -1.5) rectangle (4, 0.5);
        \Vertex[x=0,y=0,L={\jlpkg{SumOfSquares}}]{SOS}
        \SetVertexNormal[Shape=rectangle,FillColor = blue!30]
        \Vertex[x=3,y=0,L={\jlpkg{PolyJuMP}}]{PJMP}
        \SetVertexNormal[Shape=rectangle,FillColor = green!50]
        \Vertex[x=-4,y=0,L={\jlpkg{MultivariatePolynomials}}]{MP}
        \Vertex[x=-5,y=-1,L={\jlpkg{TypedPolynomials}}]{TP}
        \Vertex[x=-1.5,y=-1,L={\jlpkg{DynamicPolynomials}}]{DP}
        %\tikzset{EdgeStyle/.style={->}}
        \Edge(MP)(TP)
        \Edge(MP)(DP)
      \end{tikzpicture}
    \end{center}
  \end{frame}
  \begin{frame}[fragile]
    \frametitle{Multivariate Polynomial}
    Choose \verb|TypedPolynomials| or \verb|DynamicPolynomials|:
\begin{minted}{Julia}
using TypedPolynomials
@polyvar x y
\end{minted}
    Build a polynomial from scratch:
\begin{minted}{Julia}
using TypedPolynomials
motzkin = x^4*y^2 + x^2*y^4 + 1 - 3x^2*y^2
\end{minted}
  \end{frame}
\end{document}
