%%%%%%%%%%%%%%%%%%%%%%%%%%%%%%%%%%%%%%%%%
% Jacobs Portrait Poster
% LaTeX Template
% Version 1.0 (31/08/2015)
% (Based on Version 1.0 (29/03/13) of the landscape template
%
% Created by:
% Computational Physics and Biophysics Group, Jacobs University
% https://teamwork.jacobs-university.de:8443/confluence/display/CoPandBiG/LaTeX+Poster
%
% Further modified by:
% Nathaniel Johnston (nathaniel@njohnston.ca)
%
% Portrait version by:
% John Hammersley
%
% The landscape version of this template was downloaded from:
% http://www.LaTeXTemplates.com
%
% License:
% CC BY-NC-SA 3.0 (http://creativecommons.org/licenses/by-nc-sa/3.0/)
%
%%%%%%%%%%%%%%%%%%%%%%%%%%%%%%%%%%%%%%%%%

%----------------------------------------------------------------------------------------
%	PACKAGES AND OTHER DOCUMENT CONFIGURATIONS
%----------------------------------------------------------------------------------------

\documentclass[final]{beamer}

\usepackage[scale=1.24]{beamerposter} % Use the beamerposter package for laying out the poster

\usetheme{confposter} % Use the confposter theme supplied with this template

\setbeamercolor{block title}{fg=ngreen,bg=white} % Colors of the block titles
\setbeamercolor{block body}{fg=black,bg=white} % Colors of the body of blocks
\setbeamercolor{block alerted title}{fg=white,bg=dblue!70} % Colors of the highlighted block titles
\setbeamercolor{block alerted body}{fg=black,bg=dblue!10} % Colors of the body of highlighted blocks
% Many more colors are available for use in beamerthemeconfposter.sty

%-----------------------------------------------------------
% Define the column widths and overall poster size
% To set effective sepwid, onecolwid and twocolwid values, first choose how many columns you want and how much separation you want between columns
% In this template, the separation width chosen is 0.024 of the paper width and a 4-column layout
% onecolwid should therefore be (1-(# of columns+1)*sepwid)/# of columns e.g. (1-(4+1)*0.024)/4 = 0.22
% Set twocolwid to be (2*onecolwid)+sepwid = 0.464
% Set threecolwid to be (3*onecolwid)+2*sepwid = 0.708

\newlength{\sepwid}
\newlength{\onecolwid}
\newlength{\twocolwid}
\newlength{\threecolwid}
\setlength{\paperwidth}{36in} % A0 width: 46.8in
\setlength{\paperheight}{48in} % A0 height: 33.1in
\setlength{\sepwid}{0.024\paperwidth} % Separation width (white space) between columns
\setlength{\onecolwid}{0.3\paperwidth} % Width of one column
%\setlength{\twocolwid}{0.464\paperwidth} % Width of two columns
\setlength{\threecolwid}{0.708\paperwidth} % Width of three columns
\setlength{\topmargin}{-0.5in} % Reduce the top margin size
%-----------------------------------------------------------

\usepackage{graphicx}  % Required for including images
\graphicspath{{../../images/}}

\usepackage{booktabs} % Top and bottom rules for tables

\usepackage{minted}

\usepackage{biblatex}
\bibliography{../../biblio.bib}

\usepackage{../../custom}

%----------------------------------------------------------------------------------------
%	TITLE SECTION
%----------------------------------------------------------------------------------------

\title{Sum-of-Squares Programming in Julia with JuMP} % Poster title

\author{Beno\^it Legat$^*$, Chris Coey$^\dagger$, Robin Deits$^\dagger$, Joey Huchette$^\dagger$ and Amelia Perry$^\dagger$}

\institute{$^*$ UCLouvain, $^\dagger$ MIT} % Institution(s)

%----------------------------------------------------------------------------------------

% https://tex.stackexchange.com/questions/426088/texlive-pretest-2018-beamer-and-subfig-collide
\makeatletter
\let\@@magyar@captionfix\relax
\makeatother

\begin{document}

\addtobeamertemplate{block end}{}{\vspace*{2ex}} % White space under blocks
\addtobeamertemplate{block alerted end}{}{\vspace*{2ex}} % White space under highlighted (alert) blocks

\setlength{\belowcaptionskip}{2ex} % White space under figures
\setlength\belowdisplayshortskip{2ex} % White space under equations

\begin{frame}[t,fragile] % The whole poster is enclosed in one beamer frame

\begin{columns}[t] % The whole poster consists of three major columns, the second of which is split into two columns twice - the [t] option aligns each column's content to the top

\begin{column}{\sepwid}\end{column} % Empty spacer column

\begin{column}{\onecolwid} % The first column
  \begin{block}{Sum-of-Squares Programming}
  \begin{alertblock}{Nonnegative quadratic forms into sum of squares}
    \scalebox{0.8}{
      \begin{tikzpicture}[scale=3]
        \draw[->, line width=2pt, bend left=30] (-1, 1.6) node[left] {$(x_1, x_2, x_3)$} to (-.1, 1.3);
        \draw[->, line width=2pt, bend left=30] (-1, 1.6) to (.9, 1.25);
        \draw[->, line width=2pt, bend right=30] (2.1, 2) node[right] {\alert{unique}} to (1.55, 1.35);
        \node at (-.2, 1.2) {$p(x)$};
        \node at (0.5, 1.2) {$=$};
        \node at (1.3, 1.2) {$x^\Tr Q x$};

        \node at (-1.8,  .3) {$x_1^2 + 2x_1x_2 + 5x_2^2$};
        \node at (-1.4, -.3) {$+ 4x_2x_3 + x_3^2$};
        \node at (.1, 0) {$=$};
        \node at (1.3, 0) {$x^\Tr \begin{pmatrix}1 & 1 & 0\\1 & 5 & 2\\ 0 & 2 & 1\end{pmatrix} x$};

        \node at (-2., -1.5) {$p(x) \geq 0$ $\forall x$};
        \node at (0.5, -1.5) {$Q \succeq 0$};
        \node at (-.5, -1.5) {$\Longleftrightarrow$};
        \draw[->, line width=2pt] (1.5, -1) to node[right] {cholesky} (1.5, -2.5);

        \node at (-2, -3.2) {$(x_1 + x_2)^2 +$};
        \node at (-2, -3.8) {$(2x_2 + x_3)^2$};
        \draw[->, line width=2pt] (.2, -3.5) to (-.8, -3.5);
        \node at (2, -3.5) {$x^\Tr \begin{pmatrix}1 & 1 & 0\\0 & 2 & 1\end{pmatrix}^\Tr \begin{pmatrix}1 & 1 & 0\\0 & 2 & 1\end{pmatrix} x$};
      \end{tikzpicture}
    }
  \end{alertblock}
  \begin{alertblock}{Nonnegative polynomial into sum of squares}
    \scalebox{0.8}{
      \begin{tikzpicture}[scale=3]
        \draw[->, line width=2pt, bend left=30] (-1, 1.6) node[left] {$(x_1, x_2, x_3)$} to (-.1, 1.3);
        \draw[->, line width=2pt, bend left=20] (.6, 1.6) node[above] {$(x_1, x_1x_2, x_2)$} to (.9, 1.35);
        \draw[->, line width=2pt, bend right=30] (2.1, 2) node[right] {\alert{\emph{not} unique}} to (1.55, 1.35);
        \node at (-.1, 1.2) {$p(x)$};
        \node at (.5, 1.2) {$=$};
        \node at (1.3, 1.2) {$X^\Tr Q X$};
        \node at (-1.5, .3) {$x_1^2 + 2x_1^2x_2 + 5x_1^2x_2^2$};
        \node at (-1., -.3) {$+ 4x_1x_2^2 + x_2^2$};
        \node at (.5, 0) {$=$};
        \node at (1.8, 0) {$X^\Tr \begin{pmatrix}1 & 1 & 0\\1 & 5 & 2\\ 0 & 2 & 1\end{pmatrix} X$};

        \node at (-1.5, -1.5) {$p(x) \geq 0$ $\forall x$};
        \node at (1., -1.5) {$Q \succeq 0$};
        \node at (0., -1.5) {$\Longleftarrow$};
        \draw[->, line width=2pt] (2., -1) to node[right] {cholesky} (2., -2.5);

        \node at (-1.8, -3.2) {$(x_1 + x_1x_2)^2 +$};
        \node at (-1.8, -3.8) {$(2x_1x_2 + x_2)^2$};
        \draw[->, line width=2pt] (.2, -3.5) to (-.5, -3.5);
        \node at (2.1, -3.5) {$X^\Tr \begin{pmatrix}1 & 1 & 0\\0 & 2 & 1\end{pmatrix}^\Tr \begin{pmatrix}1 & 1 & 0\\0 & 2 & 1\end{pmatrix} X$};
      \end{tikzpicture}
    }
  \end{alertblock}

  \textbf{When is nonnegativity equivalent to sum of squares ?}

  Determining whether a polynomial is nonnegative is \alert{NP-hard}.

    \begin{description}
      \item[Hilbert 1888]
        Nonnegativity of $p(x)$ of $n$ variables and degree $2d$ is equivalent to sum of squares in the following three cases:
        \begin{itemize}
          \item $n = 1$ : Univariate polynomials
          \item $2d = 2$ : Quadratic polynomials
          \item $n = 2$, $2d = 4$ : Bivariate quartics
        \end{itemize}

      \item[Motzkin 1967]
        First explicit example:
        \[ x_1^4x_2^2 + x_1^2x_2^4 + 1 - 3x_1^2x_2^2 \geq 0 \quad \forall x \]
        but is \alert{not} a sum of squares.
        \begin{center}
          \includegraphics[trim=3cm .7cm 6cm 3cm, clip, width=\textwidth]{motzkin.png}
        \end{center}
    \end{description}

  \end{block}

\end{column}

\begin{column}{\sepwid}\end{column} % Empty spacer column

\begin{column}{\onecolwid} % The second column
  \begin{block}{Manipulating Polynomials}
    Two implementations:
    %\begin{itemize}
      %\item
        \texttt{TypedPolynomials} and
      %\item
        \texttt{DynamicPolynomials}.
    %\end{itemize}

    One common independent interface: \texttt{MultivariatePolynomials}.
\begin{minted}{Julia}
@polyvar y # one variable
@polyvar x[1:2] # tuple/vector
\end{minted}
%    Build a polynomial from scratch:
%\begin{minted}{Julia}
%motzkin = x^4*y^2 + x^2*y^4 +
%          1 - 3x^2*y^2
%\end{minted}
    Build a vector of monomials:
\begin{minted}{Julia}
X = monomials(x, 2) # [x1^2, x1*x2, x2^2]
X = monomials(x, 0:2) # [x1^2, x1*x2, x2^2, x1, x2, 1]
\end{minted}
  \end{block}

  \begin{block}{Polynomial variables}
    By hand, with an integer variable \texttt{a}:
\begin{minted}{Julia}
@variable(model, a, Int)
@variable(model, b)
p = a*x^2 + (a+b)*y^2*x + b*y^3
\end{minted}
    From a polynomial basis, e.g. the
    \emph{scaled monomial} basis,
    with integer coefficients:
\begin{minted}{Julia}
@variable(model, Poly(ScaledMonomialBasis(X)),
          Int)
\end{minted}
  \end{block}

  \begin{block}{Polynomial constraints}
    Constrain $p(x, y) \geq q(x, y)$ $\forall x, y$ such that $x \ge 0, y \ge 0, x + y \ge 1$ using the scaled monomial basis.
\begin{minted}{Julia}
S = @set x >= 0 && y >= 0 && x + y >= 1
@constraint(model, p >= q, domain = S,
            basis = ScaledMonomialBasis)
\end{minted}
    Interpreted as:
\begin{minted}{Julia}
@constraint(model, p - q in SOSCone(), domain = S,
            basis = ScaledMonomialBasis)
\end{minted}
    To use DSOS or SDSOS \cite{ahmadi2017dsos}:
\begin{minted}{Julia}
@constraint(model, p - q in DSOSCone())
@constraint(model, p - q in SDSOSCone())
\end{minted}
  \begin{alertblock}{SOS on algebraic domain}
    The domain $S$ is defined by equalities and inequalities $q_i$.
    The equalities form an \emph{algebraic variety} $V$.
    We then search for Sum-of-Squares polynomials $s_i$ such that
    \[ p(x) - q(x) \equiv s_0(x) + s_1(x) q_1(x) + \cdots \pmod{V} \]
    Groebner basis of $V$ is computed to do the division.
  \end{alertblock}
  \end{block}

  \begin{block}{Dual value}
    The dual of the constraint is a PSD matrix of moments $\mu$.
    The \texttt{extractatoms} function attempts to find an \emph{atomic} measure
    with these moments by solving an algebraic system.
  \end{block}
\end{column}

\begin{column}{\sepwid}\end{column} % Empty spacer column

\begin{column}{\onecolwid} % The third column
  %\vspace{10em}
  \begin{center}
  \begin{tikzpicture}[scale=3]
    \draw[rounded corners=20pt, fill=frambo!50] (-2, 1.5) rectangle (2, 2.5);
    \node at (0, 2) {\jlpkg{SumOfSquares}};
    \node at (0, 0) {\includegraphics{JuMP.png}};
    \draw[->, line width=1mm] (0, 1.5) to (0, 0.3);
    \draw[rounded corners=20pt] (-3, -6.5) rectangle (3, -1.5);
    \node[rotate=90] at (-2.7, -4) {\jlpkg{MathOptInterface}};
    \node[rotate=-90] at (2.7, -4) {MOI};
    \draw[rounded corners=20pt, fill=lichen!70] (-1, -2.5) rectangle (1, -3.5);
    \node at (0, -3) {Bridging};
    \draw[->, line width=1mm] (0, -.8) to (0, -2.5);
    \draw[rounded corners=20pt, fill=aurore!70] (-1, -4.5) rectangle (1, -5.5);
    \node at (0, -5) {Caching};
    \draw[->, line width=1mm] (0, -3.5) to (0, -4.5);
    \draw[->, line width=1mm] (0, -5.5) to (-2, -7.5);
    \draw[->, line width=1mm] (0, -5.5) to (-1, -7.5);
    \draw[->, line width=1mm] (0, -5.5) to ( 0, -7.5);
    \draw[->, line width=1mm] (0, -5.5) to ( 1, -7.5);
    \draw[->, line width=1mm] (0, -5.5) to ( 2, -7.5);
    \draw[rounded corners=20pt, fill=canard!60] (-3, -8.5) rectangle (3, -7.5);
    \node at (0, -8) {Solvers: Mosek, CSDP, SCS...};
%    \SetVertexNormal[Shape=rectangle,FillColor = yellow!50]
%    \draw[rounded corners=6pt] (-6.4, 2.5) rectangle (4, 3.5);
%    \Vertex[x=0,y=3,L={\jlpkg{SumOfSquares}}]{SOS}
%    \Vertex[x=0,y=1,L={\jlpkg{SumOfSquares}}]{SOS}
%
%    %\draw[->, bend left=20] (-4, 2.5) to (-1, 0.5);
%    %\draw[->, bend left=20] (-4, 2.5) to (-1, 0.5);
%    \draw[->] (-4, 2.5) .. controls (-1, 2) and (-1, 1.5) .. (-1, 0.5);
%    \draw[->] ( 3, 2.5) .. controls (-1, 2) and (-1, 1.5) .. (-1, 0.5);
%    \draw[->] ( 0, 2.5) .. controls (-1, 2) and (-1, 1.5) .. (-1, 0.5);
%
%    \draw[rounded corners=6pt] (-6.6, -1.5) rectangle (4, 0.5);
%    \Vertex[x=0,y=0,L={\jlpkg{SumOfSquares}}]{SOS}
%    \SetVertexNormal[Shape=rectangle,FillColor = blue!30]
%    \Vertex[x=3,y=0,L={\jlpkg{PolyJuMP}}]{PJMP}
%    \SetVertexNormal[Shape=rectangle,FillColor = green!50]
%    \Vertex[x=-4,y=0,L={\jlpkg{MultivariatePolynomials}}]{MP}
%    \Vertex[x=-5,y=-1,L={\jlpkg{TypedPolynomials}}]{TP}
%    \Vertex[x=-1.5,y=-1,L={\jlpkg{DynamicPolynomials}}]{DP}
%    %\tikzset{EdgeStyle/.style={->}}
%    \Edge(MP)(TP)
%    \Edge(MP)(DP)
  \end{tikzpicture}
  \end{center}
  \begin{description}
    \item[Bridging] Automatic reformulation of a constraint into an equivalent
      form supported by the solver. In particular, reformulates SOS constraints
      into PSD constraints (except \jlpkg{Alfonso} implementing \cite{papp2017sum}).
    \item[Caching] Cache of the problem data in case
      the solver do not support a modification (can be disabled).
  \end{description}

  \begin{block}{References}
    \printbibliography
  \end{block}

  %\begin{tikzpicture}
  %  \includegraphics{JuMP.png}
  %\end{tikzpicture}
\end{column}
%
%%----------------------------------------------------------------------------------------
%%	OBJECTIVES
%%----------------------------------------------------------------------------------------
%
%\begin{alertblock}{Objectives}
%
%Lorem ipsum dolor sit amet, consectetur, nunc tellus pulvinar tortor, commodo eleifend risus arcu sed odio:
%\begin{itemize}
%\item Mollis dignissim, magna augue tincidunt dolor, interdum vestibulum urna
%\item Sed aliquet luctus lectus, eget aliquet leo ullamcorper consequat. Vivamus eros sem, iaculis ut euismod non, sollicitudin vel orci.
%\item Nascetur ridiculus mus.
%\item Euismod non erat. Nam ultricies pellentesque nunc, ultrices volutpat nisl ultrices a.
%\end{itemize}
%
%\end{alertblock}
%
%%----------------------------------------------------------------------------------------
%%	INTRODUCTION
%%----------------------------------------------------------------------------------------
%
%\begin{block}{Introduction}
%
%Lorem ipsum dolor \textbf{sit amet}, consectetur adipiscing elit. Sed commodo molestie porta. Sed ultrices scelerisque sapien ac commodo. Donec ut volutpat elit. Sed laoreet accumsan mattis. Integer sapien tellus, auctor ac blandit eget, sollicitudin vitae lorem. Praesent dictum tempor pulvinar. Suspendisse potenti. Sed tincidunt varius ipsum, et porta nulla suscipit et. Etiam congue bibendum felis, ac dictum augue cursus a. \textbf{Donec} magna eros, iaculis sit amet placerat quis, laoreet id est. In ut orci purus, interdum ornare nibh. Pellentesque pulvinar, nibh ac malesuada accumsan, urna nunc convallis tortor, ac vehicula nulla tellus eget nulla. Nullam lectus tortor, \textit{consequat tempor hendrerit} quis, vestibulum in diam. Maecenas sed diam augue.
%
%Lorem ipsum dolor \textbf{sit amet}, consectetur adipiscing elit. Sed commodo molestie porta. Sed ultrices scelerisque sapien ac commodo. Donec ut volutpat elit. Sed laoreet accumsan mattis. Integer sapien tellus, auctor ac blandit eget, sollicitudin vitae lorem. Praesent dictum tempor pulvinar. Suspendisse potenti. Sed tincidunt varius ipsum, et porta nulla suscipit et. Etiam congue bibendum felis, ac dictum augue cursus a.
%
%This statement requires citation \cite{Smith:2012qr}.
%
%\end{block}
%
%%------------------------------------------------
%
%\begin{figure}
%\includegraphics[width=0.8\linewidth]{placeholder.jpg}
%\caption{Figure caption}
%\end{figure}
%
%%----------------------------------------------------------------------------------------
%
%\end{column} % End of the first column
%
%\begin{column}{\sepwid}\end{column} % Empty spacer column
%
%\begin{column}{\twocolwid} % Begin a column which is two columns wide (column 2)
%
%\begin{columns}[t,totalwidth=\twocolwid] % Split up the two columns wide column
%
%\begin{column}{\onecolwid}\vspace{-.6in} % The first column within column 2 (column 2.1)
%
%%----------------------------------------------------------------------------------------
%%	MATERIALS
%%----------------------------------------------------------------------------------------
%
%\begin{block}{Materials}
%
%The following materials were required to complete the research:
%
%\begin{itemize}
%\item Curabitur pellentesque dignissim
%\item Eu facilisis est tempus quis
%\item Duis porta consequat lorem
%\item Eu facilisis est tempus quis
%\end{itemize}
%
%The materials were prepared according to the steps outlined below:
%
%\begin{enumerate}
%\item Curabitur pellentesque dignissim
%\item Eu facilisis est tempus quis
%\item Duis porta consequat lorem
%\item Curabitur pellentesque dignissim
%\end{enumerate}
%
%\end{block}
%
%%----------------------------------------------------------------------------------------
%
%\end{column} % End of column 2.1
%
%\begin{column}{\onecolwid}\vspace{-.6in} % The second column within column 2 (column 2.2)
%
%%----------------------------------------------------------------------------------------
%%	METHODS
%%----------------------------------------------------------------------------------------
%
%\begin{block}{Methods}
%
%Lorem ipsum dolor \textbf{sit amet}, consectetur adipiscing elit. Sed laoreet accumsan mattis. Integer sapien tellus, auctor ac blandit eget, sollicitudin vitae lorem. Praesent dictum tempor pulvinar. Suspendisse potenti. Sed tincidunt varius ipsum, et porta nulla suscipit et. Etiam congue bibendum felis, ac dictum augue cursus a. \textbf{Donec} magna eros, iaculis sit amet placerat quis, laoreet id est. In ut orci purus, interdum ornare nibh. Pellentesque pulvinar, nibh ac malesuada accumsan, urna nunc convallis tortor, ac vehicula nulla tellus eget nulla. Nullam lectus tortor, \textit{consequat tempor hendrerit} quis, vestibulum in diam. Maecenas sed diam augue.
%
%\end{block}
%
%%----------------------------------------------------------------------------------------
%
%\end{column} % End of column 2.2
%
%\end{columns} % End of the split of column 2 - any content after this will now take up 2 columns width
%
%%----------------------------------------------------------------------------------------
%%	IMPORTANT RESULT
%%----------------------------------------------------------------------------------------
%
%\begin{alertblock}{Important Result}
%
%Lorem ipsum dolor \textbf{sit amet}, consectetur adipiscing elit. Sed commodo molestie porta. Sed ultrices scelerisque sapien ac commodo. Donec ut volutpat elit.
%
%\end{alertblock}
%
%%----------------------------------------------------------------------------------------
%
%\begin{columns}[t,totalwidth=\twocolwid] % Split up the two columns wide column again
%
%\begin{column}{\onecolwid} % The first column within column 2 (column 2.1)
%
%%----------------------------------------------------------------------------------------
%%	MATHEMATICAL SECTION
%%----------------------------------------------------------------------------------------
%
%\begin{block}{Mathematical Section}
%
%Nam quis odio enim, in molestie libero. Vivamus cursus mi at nulla elementum sollicitudin. Nam quis odio enim, in molestie libero. Vivamus cursus mi at nulla elementum sollicitudin.
%
%\begin{equation}
%E = mc^{2}
%\label{eqn:Einstein}
%\end{equation}
%
%Nam quis odio enim, in molestie libero. Vivamus cursus mi at nulla elementum sollicitudin. Nam quis odio enim, in molestie libero. Vivamus cursus mi at nulla elementum sollicitudin.
%
%\begin{equation}
%\cos^3 \theta =\frac{1}{4}\cos\theta+\frac{3}{4}\cos 3\theta
%\label{eq:refname}
%\end{equation}
%
%Nam quis odio enim, in molestie libero. Vivamus cursus mi at nulla elementum sollicitudin. Nam quis odio enim, in molestie libero. Vivamus cursus mi at nulla elementum sollicitudin.
%
%\begin{equation}
%\kappa =\frac{\xi}{E_{\mathrm{max}}} %\mathbb{ZNR}
%\end{equation}
%
%Nam quis odio enim, in molestie libero. Vivamus cursus mi at nulla elementum sollicitudin. Nam quis odio enim, in molestie libero. Vivamus cursus mi at nulla elementum sollicitudin.
%
%\end{block}
%
%%----------------------------------------------------------------------------------------
%
%\end{column} % End of column 2.1
%
%\begin{column}{\onecolwid} % The second column within column 2 (column 2.2)
%
%%----------------------------------------------------------------------------------------
%%	RESULTS
%%----------------------------------------------------------------------------------------
%
%\begin{block}{Results}
%
%\begin{figure}
%\includegraphics[width=0.8\linewidth]{placeholder.jpg}
%\caption{Figure caption}
%\end{figure}
%
%Nunc tempus venenatis facilisis. Curabitur suscipit consequat eros non porttitor. Sed a massa dolor, id ornare enim:
%
%Nunc tempus venenatis facilisis. Curabitur suscipit consequat eros non porttitor. Sed a massa dolor, id ornare enim:
%
%Nunc tempus venenatis facilisis. Curabitur suscipit consequat eros non porttitor. Sed a massa dolor, id ornare enim:
%
%Nunc tempus venenatis facilisis. Curabitur suscipit consequat eros non porttitor. Sed a massa dolor, id ornare enim:
%
%\begin{table}
%\vspace{2ex}
%\begin{tabular}{l l l}
%\toprule
%\textbf{Treatments} & \textbf{Res. 1} & \textbf{Res. 2}\\
%\midrule
%Treatment 1 & 0.0003262 & 0.562 \\
%Treatment 2 & 0.0015681 & 0.910 \\
%Treatment 3 & 0.0009271 & 0.296 \\
%\bottomrule
%\end{tabular}
%\caption{Table caption}
%\end{table}
%
%\end{block}
%
%%----------------------------------------------------------------------------------------
%
%\end{column} % End of column 2.2
%
%\end{columns} % End of the split of column 2
%
%\end{column} % End of the second column
%
%\begin{column}{\sepwid}\end{column} % Empty spacer column
%
%\begin{column}{\onecolwid} % The third column
%
%%----------------------------------------------------------------------------------------
%%	CONCLUSION
%%----------------------------------------------------------------------------------------
%
%\begin{block}{Conclusion}
%
%Nunc tempus venenatis facilisis. \textbf{Curabitur suscipit} consequat eros non porttitor. Sed a massa dolor, id ornare enim. Fusce quis massa dictum tortor \textbf{tincidunt mattis}. Donec quam est, lobortis quis pretium at, laoreet scelerisque lacus. Nam quis odio enim, in molestie libero. Vivamus cursus mi at \textit{nulla elementum sollicitudin}.
%
%Nunc tempus venenatis facilisis. Curabitur suscipit consequat eros non porttitor. Sed a massa dolor, id ornare enim.
%
%\end{block}
%
%%----------------------------------------------------------------------------------------
%%	ADDITIONAL INFORMATION
%%----------------------------------------------------------------------------------------
%
%\begin{block}{Additional Information}
%
%Maecenas ultricies feugiat velit non mattis. Fusce tempus arcu id ligula varius dictum.
%\begin{itemize}
%\item Curabitur pellentesque dignissim
%\item Eu facilisis est tempus quis
%\item Duis porta consequat lorem
%\end{itemize}
%
%Maecenas ultricies feugiat velit non mattis. Fusce tempus arcu id ligula varius dictum.
%\begin{itemize}
%\item Curabitur pellentesque dignissim
%\item Eu facilisis est tempus quis
%\item Duis porta consequat lorem
%\end{itemize}
%
%\end{block}
%
%%----------------------------------------------------------------------------------------
%%	REFERENCES
%%----------------------------------------------------------------------------------------
%
%\begin{block}{References}
%
%\nocite{*} % Insert publications even if they are not cited in the poster
%\small{\bibliographystyle{unsrt}
%\bibliography{sample}\vspace{0.75in}}
%
%\end{block}
%
%%----------------------------------------------------------------------------------------
%%	ACKNOWLEDGEMENTS
%%----------------------------------------------------------------------------------------
%
%\setbeamercolor{block title}{fg=red,bg=white} % Change the block title color
%
%\begin{block}{Acknowledgements}
%
%\small{\rmfamily{Nam mollis tristique neque eu luctus. Suspendisse rutrum congue nisi sed convallis. Aenean id neque dolor. Pellentesque habitant morbi tristique senectus et netus et malesuada fames ac turpis egestas.}} \\
%
%\end{block}
%
%%----------------------------------------------------------------------------------------
%%	CONTACT INFORMATION
%%----------------------------------------------------------------------------------------
%
%\setbeamercolor{block alerted title}{fg=black,bg=norange} % Change the alert block title colors
%\setbeamercolor{block alerted body}{fg=black,bg=white} % Change the alert block body colors
%
%\begin{alertblock}{Contact Information}
%
%\begin{itemize}
%\item Web: \href{http://www.my.edu/smithlab}{http://www.my.edu/smithlab}
%\item Email: \href{mailto:john@smith.com}{john@smith.com}
%\item Phone: +1 (000) 111 1111
%\end{itemize}
%
%\end{alertblock}
%
%\begin{center}
%\begin{tabular}{ccc}
%\includegraphics[width=0.4\linewidth]{logo.png} & \hfill & \includegraphics[width=0.4\linewidth]{logo.png}
%\end{tabular}
%\end{center}
%
%%----------------------------------------------------------------------------------------
%
%\end{column} % End of the third column

\end{columns} % End of all the columns in the poster

\end{frame}

\end{document}
