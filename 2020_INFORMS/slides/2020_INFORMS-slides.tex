\documentclass{beamer}
\usetheme{metropolis}
%\setbeamersize{text margin left=.2cm,text margin right=.2cm}
\usepackage{graphicx}
\graphicspath{{../../images/}}
%\usepackage[french]{babel}
%\usepackage{listings}
%\usepackage{lipsum}
\usepackage{boolexpr}
\usepackage{kpfonts}
\usepackage{caption}
\usepackage{wrapfig}
%\usepackage{chngcntr}
\usepackage[labelformat=empty]{caption}
\usepackage[official]{eurosym}

\usepackage{minted}

\usepackage{siunitx}

% http://tex.stackexchange.com/questions/114830/how-can-i-use-lvert-and-rvert-norm-symbols-x-with-the-iwona-math-font
\usepackage[math]{iwona}
\usepackage{scalerel}
\def\lVert{\mid\!\mid}
\def\rVert{\mid\!\mid}

\usepackage[normalem]{ulem}
%\newcommand{\Adj}{\mathbf{A}}
\usepackage{mathtools}

\usepackage{../../custom}
\usepackage{amsfonts}
%\newcommand{\jsrcodepath}{../../code}
%\usepackage{jsr}

\newcommand{\expe}[2]{\la #1, #2 \ra}

\usepackage{framed}

%\usepackage{mathtools,xparse}
%\DeclarePairedDelimiter{\norm}{\lVert}{\rVert}
\newcommand\Wider[2][3em]{%
\makebox[\linewidth][c]{%
  \begin{minipage}{\dimexpr\textwidth+#1\relax}
  \raggedright#2
  \end{minipage}%
  }%
}

\newcommand{\aeur}{\alpha_\text{\euro}}
\newcommand{\adol}{\alpha_\$}
\newcommand{\apou}{\alpha_\text{\pounds}}

\title{SumOfSquares: A Julia package for Polynomial Optimization}
\date{November, 2020}
\author{Beno\^it Legat, Tillmann Weisser}
%\institute{Los Alamos National Laboratory, 16th July 2019}
%\institute{UCLouvain\\
%           Massachusetts Institute of Technology (MIT)\\
%           Rice University\\
%           Los Alamos National Laboratory (LANL)}

% https://tex.stackexchange.com/questions/426088/texlive-pretest-2018-beamer-and-subfig-collide
\makeatletter
\let\@@magyar@captionfix\relax
\makeatother
\begin{document}
  \maketitle

\begin{frame}
  \frametitle{History}
  \begin{itemize}
    \item Started in August 2016: \texttt{SOSTOOLS} demos.
    \item October 2016: Amelia Perry, Joey Huchette, Robin Deits and Chris Coey join,
      package split into \texttt{MultivariatePolynomials}, \texttt{PolyJuMP} and \texttt{SumOfSquares}.
    \item April 2017, \texttt{MultivariatePolynomials} split into
      \texttt{MultivariatePolynomials}, \texttt{DynamicPolynomials}, \texttt{TypedPolynomials},
    \item August 2017 \texttt{MultivariateMoments} and \texttt{SemialgebraicSets} moved out of \texttt{SumOfSquares}.
    \item June 2017--March 2019 Development of \texttt{MathOptInterface} (MOI) $\to$ rewrite of \texttt{JuMP}, \texttt{PolyJuMP} and \texttt{SumOfSquares}.
    \item July 2019--December 2020 Development of \texttt{MutableArithmetics} for \alert{performance}.
    \item 2019--2020 \texttt{MomentOpt} and Tillmann Tweisser push for more \alert{features}.
  \end{itemize}
\end{frame}

\begin{frame}{MutableArithmetics}
  What is mutable in (JuMP/MOI var: $\alpha$, Polynomial var: $x$):
  \[
    A = \begin{bmatrix}
      \alpha x + 1 & x - \alpha\\
      x - \alpha & x + 2
    \end{bmatrix}
  \]
  \uncover<2>{
    3 nested levels of mutability (4 with e.g. \texttt{BigInt}/\texttt{BigFloat}) $\to$
    huge toll on performance.

    MutableArithmetics API:
    \begin{itemize}
      %\item \texttt{MA.operate}: Output is safe to mutate.
      \item \texttt{MA.operate!} / \texttt{MA.operate\_to!}: May mutate first argument.
      \item \texttt{MA.mutable\_operate!} / \texttt{MA.mutable\_operate\_to!}: Must mutate first argument.
    \end{itemize}
  }
\end{frame}

\begin{frame}{Polynomial basis}
  \[ p(x) = b(x)^\Tr Q b(x) \]
  Basis $b$ from \texttt{MultivariateBases} package:
  \begin{itemize}
    \item \texttt{MonomialBasis}: $x^2, xy, y^2$
    \item \texttt{ScaledMonomialBasis}: $x^2, \sqrt{2} xy, y^2$
    \item Orthogonal basis: Hermite, Laguerre, Legendre and Chebyshev bases.
    \item More to come... (e.g. Lagrange interpolation basis)
  \end{itemize}
\end{frame}

\begin{frame}{Certificates}

  \emph{Ideal} certificates (SOS on algebraic variety):
  \begin{itemize}
    \item Remainder: $p(x)$ SOS when $x \in S$ $\to$ remainder modulo $S$.
    \item Newton polytope.
    \item Maxdegree (all monomials up to degree $d$).
  \end{itemize}
  \emph{Preorder} certificates (SOS on basic semialgebraic set):
  \begin{itemize}
    \item Putinar certificate:
      \[ g_i(x) \ge 0 \Rightarrow p(x) \ge 0 \quad \leftrightarrow \quad p(x) = s_0(x) + \sum_i s_i g_i. \]
  \end{itemize}

  \alert{Flexibility} by combining existing or creating new certificates.
\end{frame}

\begin{frame}{Sparsity reduction}
  \[ p(x) = b(x)^\Tr Q b(x) \to p(x) = b_1(x)^\Tr Q_1 b_1(x) + \cdots + b_k(x)^\Tr Q_k b_k(x) \]

  \begin{itemize}
    \item \texttt{SignSymmetry} certificate.
    \item \alert{Chordal} sparsity certificates:
      \begin{itemize}
        \item \texttt{VariableSparsity}: Variable/correlative sparsity.
        \item \texttt{MonomialSparsity}: Monomial/term sparsity.
      \end{itemize}
      \alert{Chordal} \emph{extension} implementation with \emph{greedy} fill-in heuristics.
      Should move to separate package (with \texttt{COSMO} ?).
  \end{itemize}
\end{frame}

\begin{frame}{Sums of Hermitian squares / non-commutative}
  \[ p(x) = b(x)^\Tr Q b(x) \]
  with $x_1x_2 \neq x_2x_1$, $Q \in \mathbb{C}^{n \times n}$ hermitian PSD:
  \begin{itemize}
    \item Initial work on extending MOI to \emph{complex} numbers in \texttt{ComplexOptInterface},
      e.g. \begin{center} \alert{complex} hermitian PSD $\to$ \alert{real} symmetric PSD. \end{center}
    \item \emph{Non-commutative} algebra implemented in \texttt{DynamicPolynomials},
      i.e. created with \texttt{@ncpolyvar}.
    \item Implementation of Newton polytope for non-commutative variables.
    \item Still features missing compared to \texttt{NCSOStools},
      e.g. atom extraction with \texttt{MultivariateMoments}.
  \end{itemize}
\end{frame}

\begin{frame}{Future work}
  \begin{itemize}
    \item Polynomial basis for equality (Pull Requests opened by Tillmann Weisser).
    \item Symmetry reduction (Good progress made by Tillmann Weisser and Marek Kaluba).
    \item Certificates for \texttt{PolyJuMP}: equality in ideal without Groebner basis.
    \item Better integration with \texttt{Hypatia} (just released!)
    \item More certificates: Schm\"{u}dgen, ...
    \item Rework examples with Literate to include in documentation.
  \end{itemize}
\end{frame}

\end{document}
