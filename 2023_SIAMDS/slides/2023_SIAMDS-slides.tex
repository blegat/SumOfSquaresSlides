\documentclass{beamer}
\usetheme[block=fill]{metropolis}
%\setbeamersize{text margin left=.2cm,text margin right=.2cm}
\usepackage{graphicx}
\graphicspath{{../../images/}}
%\usepackage[french]{babel}
%\usepackage{listings}
%\usepackage{lipsum}


\usepackage{kpfonts}
\usepackage[style=alphabetic,backend=bibtex]{biblatex}
\addbibresource{../../biblio.bib}
%\usepackage{boolexpr}
%\usepackage{caption}
%\usepackage{wrapfig}
%\usepackage{chngcntr}
%\usepackage[labelformat=empty]{caption}
%\usepackage[official]{eurosym}

\newcommand{\newtonpoly}[1]{\mathcal{N}(#1)}
\newcommand{\supfun}[2]{\delta^*(#1|#2)}

\renewcommand*{\footnotesize}{\tiny}

% Citations in footnote
% https://tex.stackexchange.com/a/29931/38244

\DeclareCiteCommand{\footpartcite}[\mkbibfootnote]
{\usebibmacro{prenote}}%
{\usebibmacro{citeindex}%
    \mkbibbrackets{\usebibmacro{cite}}%
    \setunit{\addnbspace}
    \printnames{labelname}%
    \setunit{\labelnamepunct}
    \printfield[citetitle]{title}%
    \newunit
    \printfield[]{year}}
{\addsemicolon\space}
{\usebibmacro{postnote}}

\setbeamertemplate{footnote}{\insertfootnotetext}

\newcommand{\induced}{\rho}
\newcommand{\induceda}{\induced^{(1)}}
\newcommand{\inducedb}{\induced^{(2)}}

%\newcommand{\gatpar}{Gatermann, Parrilo, \emph{{S}ymmetry groups, semidefinite programs, and sums of squares}, Journal of Pure and Applied Algebra, 2004}
\newcommand{\gatpar}{Gatermann, Parrilo, \emph{{S}ymmetry groups, semidefinite programs, and sums of squares}, 2004}
\newcommand{\citefoot}[1]{\footnote{\tiny #1}}

\usepackage{siunitx}

% http://tex.stackexchange.com/questions/114830/how-can-i-use-lvert-and-rvert-norm-symbols-x-with-the-iwona-math-font
\usepackage[math]{iwona}
\usepackage{scalerel}
\def\lVert{\mid\!\mid}
\def\rVert{\mid\!\mid}

\usepackage[normalem]{ulem}
%\newcommand{\Adj}{\mathbf{A}}
\usepackage{mathtools}

\usepackage{../../custom}
\usepackage{amsfonts}
%\newcommand{\jsrcodepath}{../../code}
%\usepackage{jsr}

\newcommand{\expe}[2]{\la #1, #2 \ra}

\usepackage{framed}

%\usepackage{mathtools,xparse}
%\DeclarePairedDelimiter{\norm}{\lVert}{\rVert}
\newcommand\Wider[2][3em]{%
\makebox[\linewidth][c]{%
  \begin{minipage}{\dimexpr\textwidth+#1\relax}
  \raggedright#2
  \end{minipage}%
  }%
}

\newcommand{\aeur}{\alpha_\text{\euro}}
\newcommand{\adol}{\alpha_\$}
\newcommand{\apou}{\alpha_\text{\pounds}}

\title{Exploiting the Structure of a Polynomial Optimization Problem}
\date{May 14th, 2023}
\author{Beno\^it Legat}
\institute{SIAM Dynamical Systems 2023}
%\institute{UCLouvain\\
%           Massachusetts Institute of Technology (MIT)\\
%           Rice University\\
%           Los Alamos National Laboratory (LANL)}

% https://tex.stackexchange.com/questions/426088/texlive-pretest-2018-beamer-and-subfig-collide
\makeatletter
\let\@@magyar@captionfix\relax
\makeatother
\begin{document}
  \maketitle

\begin{frame}{Table of contents}
    \tableofcontents
\end{frame}

\section{Challenges}

\begin{frame}{Sum-of-Squares Programming}
    \begin{block}{It may look simple at first...}
        $p(x)$ is SOS $\to$
        $p(x) = b(x)^\Tr Q b(x)$
        where $Q$ is PSD.
    \end{block}
    \begin{itemize}
      \item PSD $\to$ (scaled) diagonally dominant ? \emph{DSOS/SDSOS}
      \item \textbf{reformulation as geometric or standard conic form ?}
      \item \textbf{what polynomial space for $b(x)$ ? \emph{Newton polytope}}
      \item which basis for $b(x)$ ?
      \item which basis for $p - b^\Tr Q b$ ? \alert{Ill-conditioned} change of basis ?
      \item any group symmetry ? Can we reduce \alert{symbolically} ?
      \item \emph{Chordal} sparsity ? \emph{Term} sparsity ? \emph{Sign} symmetry ?
      \item extract roots of $p(x)$ from dual \alert{moment matrix} ?
      \item Different formulation ? \alert{Hypatia}/\alert{Alphone}, Burer-Monteiro ?
    \end{itemize}
\end{frame}

\begin{frame}{With constraints}
    $$p(x) = s_0(x) + \sum \lambda_i(x) h_i(x) + \sum s_i(x) g_i(x), \quad s_i(x) = b_i(x)^\Tr Q_i b_i(x)$$
    \begin{itemize}
      \item Explicit $\lambda_i$ or remainder with Gr\"{o}bner basis ?
      \item Putinar or Sch\"{u}dgen certificate ?
      \item \textbf{what polynomial space for $b_i(x)$ ? \emph{Newton polytope}}
      \item which basis for $b_i(x)$ ?
      \item ...
    \end{itemize}
\end{frame}

\section{Geometric or standard form}

\begin{frame}{Geometric and standard form}
    \emph{Standard conic form} SDP:
    \begin{equation*}
          \begin{array}{rl}
              \min\limits_{Q \in \SymK[n]} & \la C, Q \ra\\
              \text{subject to:} & \la A_i, Q \ra = b_i, \quad i=1,2,\ldots,m\\
                                 & Q \succeq 0,
          \end{array}
      \end{equation*}
      \emph{Geometric conic form} SDP:
      \begin{equation*}
          \begin{array}{rl}
              \max\limits_{y \in \mathbb{R}^m} & \la b, y \ra\\
              \text{subject to:} & C \succeq \sum_{i=1}^m A_i y_i\\
                                 & y\ \mathsf{free},
          \end{array}
      \end{equation*}
    \begin{align*}
    \end{align*}
\end{frame}

\begin{frame}{Standard conic form}
    \begin{block}{Notation}
        \begin{align*}
            p(x) & = \sum_{\alpha} p_\alpha x^\alpha &
            \mathcal{A}_\alpha &= \setbuild{(\beta, \gamma) \in b^2}{x^\beta x^\gamma = x^\alpha}
        \end{align*}
    \end{block}
    \begin{block}{Standard conic form}
        \begin{align*}
            \la \sum_{(\beta, \gamma) \in \mathcal{A}_\alpha} e_\beta e_\gamma^\Tr, Q \ra & = p_\alpha, \quad\forall \alpha\\
            Q & \succeq 0
        \end{align*}
    \end{block}
\end{frame}

\begin{frame}{Geometric conic form}
    \begin{block}{Notation}
        \begin{align*}
            p(x) & = \sum_{\alpha} p_\alpha x^\alpha &
            \mathcal{A}_\alpha &= \setbuild{(\beta, \gamma) \in b^2}{x^\beta x^\gamma = x^\alpha}
        \end{align*}
        Let $(\beta_\alpha, \gamma_\alpha) \in \mathcal{A}_\alpha$.
    \end{block}
    \begin{block}{Geometric conic form}
        \begin{align*}
            \sum_\alpha
            p_\alpha e_{\beta_\alpha} e_{\gamma_\alpha}^\Tr +
            \sum_{(\beta, \gamma) \in \mathcal{A}_\alpha \setminus (\beta_\alpha, \gamma_\alpha)}
            y_{\beta,\gamma} (e_\beta e_\gamma - e_{\beta_\alpha} e_{\gamma_\alpha}^\Tr)^\Tr
            & \succeq 0\\
            y_{\beta,\gamma} & \text{ free}
        \end{align*}
    \end{block}
\end{frame}

\begin{frame}{Which one do I choose ?}
    \begin{itemize}
        \item \alert{Standard} conic form is good when \alert{low} number of \alert{variables} and \alert{high degree}.
        Univariate $2d$: standard gives \alert{linear} $m = 2d + 1$
        and geometric gives \alert{quadratic} $m = d(d + 1)/2 - (2d + 1)$.
        \item \alert{Geometric} conic form is good when \alert{high} number of \alert{variables} and \alert{low degree}.
        Quadratic: standard gives \alert{quadratic} $m$ and geometric gives $m = 0$.
    \end{itemize}

    What's the \alert{threshold} used in practice ?
    None, \alert{user} chooses !

    \uncover<2>{
        In \textsc{yalmip}, \texttt{sosmodel} uses geometric and \texttt{solvesos} uses standard.

        In \jlpkg{SumOfSquares}, formulation matches \alert{solver}'s conic form !
        Hence the importance of playing with \alert{\jlpkg{Dualization}} !
    }
\end{frame}

\section{Newton polytope}

\begin{frame}{Newton polytope}
    \begin{block}{Notation}
        \begin{align*}
            p(x) & = \sum_{\alpha} p_\alpha x^\alpha &
            \mathcal{A}_\alpha &= \setbuild{(\beta, \gamma)}{x^\beta x^\gamma = x^\alpha}
        \end{align*}
    \end{block}
    \begin{block}{Key observation 1}
        If $\mathcal{A}_\alpha = \{(\beta,\beta)\}$ and $p_\alpha = 0$ then $Q_{\beta,\beta} = 0$.
    \end{block}
    \begin{block}{Key observation 2}
        If $Q_{\beta,\beta} = 0$ then $Q_{\beta,\gamma} = Q_{\gamma,\beta} = 0$, $\forall \gamma$.
    \end{block}
\end{frame}



\begin{frame}{Newton chip method}
  %\begin{algorithm}[H]
  %  \caption{Newton chip method}
    \begin{algorithmic}
      \WHILE{$\exists \alpha, \beta$ s.t. $\mathcal{A}_\alpha = \{(\beta,\beta)\}, p_\alpha = 0$}
        \STATE remove $\beta$ from $b$
      \ENDWHILE
    \end{algorithmic}
  %\end{algorithm}
    \begin{itemize}
      \item Easy to implement
      \item Easy to generalize, e.g., \autocite[Section~2.3]{burgdorf2016optimization}\footpartcite{burgdorf2016optimization}
      \item Costly : \alert{quadratic} in length of $b$ $\to$ best trim down $b$ before
      \item \alert{Signed} variant: change $p_\alpha \alert{=} 0$ into $p_\alpha \alert{\le} 0$
    \end{itemize}
\end{frame}

\begin{frame}{Newton polytope}
    \begin{block}{Polytopic reformulation of key observation 2}
        $\newtonpoly{b} + \newtonpoly{b} = 2\newtonpoly{b}$ where $+$ is Minkwoski sum.
    \end{block}
    So the result of the commucative unsigned newton chip method is:
    $$\newtonpoly{b} = \lfloor \mathcal{N}(p) / 2 \rfloor$$
    \vspace{-2em}
    \begin{itemize}
        \item Less costly because not quadratic in length of $b$
        \item How to enumerate the elements given V-rep of $\lfloor \mathcal{N}(p) / 2 \rfloor$ ?
        \begin{itemize}
            \item solve LP's~\autocite{Lofberg2009}\footpartcite{Lofberg2009}
            \item Compute H-rep with CDD/LRS/PPL~\autocite{prajna2004new}\footpartcite{prajna2004new}
        \end{itemize}
    \end{itemize}
\end{frame}

\begin{frame}{Wait, isn't polyhedral computation not cheap ?}
    If polyhedral computation is costly, we might as well just do Newton chip method!

    We can at least compute a cheap outer approximation of
    $$\lfloor \mathcal{N}(p) / 2 \rfloor$$
    \begin{block}{Outer-approximation of \autocite{prajna2004new}\footpartcite{prajna2004new}}
        \begin{itemize}
            \item Min and max total degree
            \item Min and max degree of each variable
            \item Min and max degree groups (\emph{multipartite})
        \end{itemize}
    \end{block}
\end{frame}

\begin{frame}{Support function interpretation}
    $2\newtonpoly{b} \subseteq \newtonpoly{p}$ is equivalent to
    \[ \forall y \in \R^n, 2\supfun{y}{\newtonpoly{b}} \le \supfun{y}{\newtonpoly{p}}. \]
    Generalizes previous cases:
    \begin{itemize}
        \item Max total degree: $y = \mathbf{1}$
        \item Min total degree: $y = -\mathbf{1}$
        \item Max degree of $x_i$: $y = e_i$
        \item Min degree of $x_i$: $y = -e_i$
        \item Max degree group $\{x_{i_1}, \ldots, x_{i_j}\}$, $y = e_{i_1} + \cdots + e_{i_j}$
        \item Min degree group $\{x_{i_1}, \ldots, x_{i_j}\}$, $y = -e_{i_1} - \cdots - e_{i_j}$
    \end{itemize}
\end{frame}

\begin{frame}{Signed Newton chip for Putinar}
    \begin{block}{Notation}
        \begin{align*}
            g_0 & = 1 &
            \mathcal{A}_{\alpha}^i &= \setbuild{(\beta, \gamma) \in b_i^2}{x^\beta x^\gamma = x^\alpha}
        \end{align*}
    \end{block}
    \begin{block}{Generalized key observation 1}
        If
        \[ \not\exists i, \beta \neq \gamma \in b_i, \delta \in \newtonpoly{g_i} \text{ s.t. } x^\delta x^\gamma x^\beta = x^\alpha \]
        and
        \[ \setbuild{g_{i,\delta}}{\exists \beta_i \in b_i \text{ s.t. } x^\delta x^{2\beta_i} = x^\alpha} \]
        has constant sign and $p_\alpha$ is zero or has opposite sign,
        then $Q_{i,\beta_i,\beta_i} = 0$
    \end{block}
\end{frame}

\begin{frame}{Support function for Putinar}
    \[ \forall y \in \R^n, \supfun{y}{\newtonpoly{p}} = \supfun{y}{\newtonpoly{\sum g_i s_i}}. \]
    Could be \alert{cancellations} if \alert{negative} coefficients in $g_i$
    \[ \forall y \in \R^n, \supfun{y}{\newtonpoly{p}} \subseteq \supfun{y}{\sum \newtonpoly{g_i} + \newtonpoly{s_i}}. \]
    \alert{Support} function on \alert{Minkowski} sum:
    \begin{align*}
      \supfun{y}{\mathcal{S} + \mathcal{T}} & = \supfun{y}{\mathcal{S}} + \supfun{y}{\mathcal{T}}\\
      \forall y \in \R^n, \supfun{y}{\newtonpoly{p}} & \subseteq \sum \supfun{y}{\newtonpoly{g_i}} + \supfun{y}{\newtonpoly{s_i}}\\
                                                     & \subseteq \sum \supfun{y}{\newtonpoly{g_i}} + 2\supfun{y}{\newtonpoly{b_i}}
    \end{align*}
    If support function uniquely maximized by \alert{diagonal} elements $Q_{i_1}, \ldots, Q_{i_j}$
    with same signs $g_{i_1}, \ldots, g_{i_j}$.
    If $\supfun{y}{\newtonpoly{p}}$ \alert{strictly} smaller or different sign, \alert{reduce} by removing $b_{i_1}, \ldots, b_{i_j}$.
\end{frame}

\section{Conclusion}

\begin{frame}{Newton polytope vs sparsity/symmetry}
    \begin{itemize}
        \item Newton polytope \alert{reduces} the basis $b$
        \item Sparsity/symmetry try to block diagonalize $b$, but $b$ is \alert{not reduced},
    \end{itemize}
    \alert{Complementary} reductions, a good Newton polytope reduction is as important as sparsity/symmetry reduction.
\end{frame}

\begin{frame}{Conclusion}
    \begin{itemize}
        \item Always try to solve your problem with and without \jlpkg{Dualization}
        \item Specify constraints with \texttt{domain} keyword to get Newton polytope-reduced bases.
        \item Implemented in \jlpkg{SumOfSquares}, feedback is welcome!
    \end{itemize}
\end{frame}

\begin{frame}{Algebraic geometry}
    \begin{itemize}
        \item Interface \jlpkg{MultivariatePolynomials} with implementations
            \begin{itemize}
                \item \jlpkg{DynamicPolynomials}
                \item \jlpkg{TypedPolynomials}
                \item \jlpkg{SIMDPolynomials}
                \item \jlpkg{CondensedMatterSOS}
            \end{itemize}
        \item Polynomials bases in \jlpkg{MultivariateBases}
        \item Gröbner bases and algebraic system solving in \jlpkg{SemialgebraicSets}.
            Interface with Buchberger and multiplication matrices by default
            with other implementations:
            \begin{itemize}
                \item \jlpkg{HomotopyContinuation}
                \item \jlpkg{Groebner}
            \end{itemize}
        \item Extract roots from moment matrix with \jlpkg{MultivariateMoments}
    \end{itemize}
\end{frame}

\begin{frame}{Optimization with JuMP}
    \begin{columns}
    \begin{column}{0.9\textwidth}
    \begin{itemize}
        \item Solver interface \jlpkg{MathOptInterface} implemented by (SDP-only):
        \begin{itemize}
            \item Nonsymmetric cone interior point: \jlpkg{Hypatia}
            \item \jlpkg{MosekTools}
            \item ADMM: \jlpkg{SCS}, \jlpkg{COSMO}
            \item MATLAB: \jlpkg{SeDuMi}, \jlpkg{SDPNAL}, \jlpkg{SDPT3}
            \item C/C++: \jlpkg{CSDP}, \jlpkg{SDPA}, \jlpkg{DSDP}
            \item Burer-Monteiro: \jlpkg{SDPLR}
            \item BMI, NLSDP: \jlpkg{Penopt}
        \end{itemize}
        \item \jlpkg{JuMP} extension for optimization with polynomials \jlpkg{PolyJuMP}
          \begin{itemize}
            \item Solve KKT system with \jlpkg{SemialgebraicSets}
            \item Sums of AM/GM Exponential (SAGE) : \url{https://github.com/jump-dev/SumOfSquares.jl/pull/240/}
            \item Sum-of-Squares with \jlpkg{SumOfSquares}
          \end{itemize}
    \end{itemize}
    \end{column}
    \begin{column}{0.1\textwidth}
    \includegraphics[scale=0.11]{JuMP.png}
    \end{column}
    \end{columns}
\end{frame}


\end{document}
